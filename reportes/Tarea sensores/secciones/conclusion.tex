\section{Conclusión} \label{sec:conclusion}
Los sensores internos y externos son fundamentales en la recopilación de datos y el monitoreo de sistemas, diferenciándose por su ubicación y las variables que miden. Los sensores internos se encuentran dentro de un dispositivo o sistema, supervisando condiciones internas como temperatura, presión o voltaje, y son clave para asegurar el correcto funcionamiento y estabilidad de equipos, como motores o baterías. Por otro lado, los sensores externos están ubicados fuera del sistema y se utilizan para detectar estímulos en el entorno, como proximidad, movimiento o gases, y son esenciales en aplicaciones como seguridad, monitoreo ambiental o control de procesos industriales. Ambos tipos trabajan en conjunto para crear sistemas más eficientes, inteligentes y seguros, mejorando la interacción con el entorno y optimizando el rendimiento.
