\section{Cinemática} \label{sec:cinematica}

Explicar su definición en base a lo que viene en libros o de los archivos de Matlab. Diferenciar entre \textbf{Cinemática directa} y \textbf{Cinemática inversa}.
Para el proceso sobre cómo se llevó a cabo el algoritmo en Matlab, ir a la  \hyperref[sec:proceso_cinematica]{sección del proceso de Cinemática}.

\subsection{Cinemática Directa}
Explicar sobre la definición geométrica usando Denavit Hartenberg con el algoritmo y las matemáticas.

\subsection{Cinemática Diferencial}
Aquí explicarán sobre el jacobiano para la cinemática diferencial.

\subsection{Cinemática Inversa}
Explicar su definición y cómo se relaciona el gradiente con el jacobiano. Cabe destacar que existen multitud de métodos y en mi clase solo usaron un método, pero aquí pueden explicar las imágenes que vienen en Matlab.

