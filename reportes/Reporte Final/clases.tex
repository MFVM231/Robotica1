\usepackage[skins,minted]{tcolorbox}
\usepackage[utf8]{inputenc}
\usepackage[T1]{fontenc}
\usepackage[spanish, es-tabla]{babel}
\languageshorthands{spanish}
%\usepackage{lmodern} % Usa las fuentes modernas de LaTeX
\usepackage[numbers]{natbib}
%\usepackage{morewrites}
\usepackage{inconsolata}
\usepackage{mdframed}
\usepackage{minted}
\usepackage{bm}
\usepackage{listingsutf8}
\usepackage{subcaption}
\usepackage{amsmath}
\usepackage{amssymb}
\usepackage{graphicx}
\usepackage{hyperref}
\usepackage{longtable}
\usepackage{tabularx}
\usepackage{threeparttable}
\usepackage{array}    % Necesario para centrar horizontal y verticalmente
\usepackage{xcolor}
\usepackage{pdfpages}
%\usepackage{color}
\usepackage{fancyhdr}
\usepackage{menukeys}
\usepackage{appendix}
\usepackage{fontawesome}
\usepackage{comment}
\usepackage{caption}
\usepackage{setspace}
\usepackage[explicit]{titlesec}
\usepackage{booktabs}
\usepackage[a4paper,margin=2cm]{geometry}

%\geometry{top=3cm, bottom=3cm, left=4cm, right=2cm} % Si lo quieres imprimir

\definecolor{green1}{HTML}{1dae28}
\definecolor{green2}{HTML}{afd095}
\definecolor{lightgray}{gray}{0.9}
\definecolor{orange}{RGB}{18,84,183}
\definecolor{titulo}{gray}{0.75}
\definecolor{gray97}{gray}{.97}
\definecolor{gray75}{gray}{.75}
\definecolor{gray45}{gray}{.45}
\definecolor{advertencia}{RGB}{255,178,102}
\definecolor{colorturqueza}{RGB}{178,223,238}
\definecolor{mintedbackground}{rgb}{0.95,0.95,0.95}
\definecolor{lbcolor}{rgb}{0.95,0.95,0.95}
\definecolor{mintedframe}{rgb}{0.0,0.0,0.0}
\definecolor{codebg}{rgb}{0.96,0.96,0.96}
\definecolor{colorurls}{RGB}{107,17,17}
\definecolor{colorsql}{RGB}{255,245,245}
\definecolor{colorreferences}{RGB}{48,134,3}
\definecolor{titulo}{gray}{0.65}			%------ color para fondo del titulo de tablas.

\hypersetup{
	%bookmarks=true,         % show bookmarks bar?
	unicode=true,          % non-Latin characters in Acrobat’s bookmarks
	pdftoolbar=true,        % show Acrobat’s toolbar?
	pdfmenubar=true,        % show Acrobat’s menu?
	pdffitwindow=false,     % window fit to page when opened
	pdfstartview={FitH},    % fits the width of the page to the window
	pdftitle={Reporte final de Robótica},    % title
	%	pdfauthor={},     % author
	pdfsubject={Reporte final de Robótica},   % subject of the document
	%pdfcreator={pdfTeX 3.14159265-2.6-1.40.16 (TeX Live 2016/dev)},   % creator of the document
	%pdfproducer={Panel HJ 2017}, % producer of the document
	pdfkeywords={Manipulador industrial} {Robotica} {ros} {gazebo}, % list of keywords
	%pdfnewwindow=true,      % links in new PDF window
	colorlinks=true,       % false: boxed links; true: colored links
	linkcolor=black,          % color of internal links (change box color with linkbordercolor)
	citecolor=colorreferences,        % color of links to bibliography
	filecolor=magenta,      % color of file links
	urlcolor=blue,           % color of external links
	linkbordercolor={0 0 0}
}

\lstset{
	inputencoding=utf8,
	language=Python,
	frame=Ltb,
	tabsize=2,
	framerule=0pt,
	aboveskip=0.5cm,
	framextopmargin=0pt,
	framexbottommargin=0pt,
	framexleftmargin=0.4cm,
	framesep=0pt,
	rulesep=.0pt,
	backgroundcolor=\color{gray97},
	rulesepcolor=\color{blue},
	%
	stringstyle=\ttfamily,
	showstringspaces = false,
	basicstyle=\small\ttfamily,
	commentstyle=\color{gray45},
	keywordstyle=\bfseries,
	%
	numbers=none,
	numbersep=15pt,
	numberstyle=\tiny,
	numberfirstline = false,
	breaklines=true
}

\setminted[matlab]{%
	breaklines=true,       % activa el ajuste de línea
	breakanywhere=true,    % permite partir en cualquier punto si no hay espacios
	breaksymbolleft=,      % quita el símbolo de continuación
	autogobble            % elimina sangrías comunes
%	fontsize=\footnotesize % reduce tamaño de fuente para que quepa mejor
}

\setminted[bash]{
	bgcolor=mintedbackground,
	fontfamily=tt,
	linenos=true,
	numberblanklines=true,
	numbersep=11pt,
	numbersep=2pt,
	gobble=0,
	frame=leftline,
	framesep=2mm,
	funcnamehighlighting=false,
	tabsize=4,
	obeytabs=false,
	samepage=false,
	showspaces=false,
	showtabs =false,
	texcl=false,
	baselinestretch=1.2,
	fontsize=\footnotesize,
	breaklines=true,
	breaksymbolleft=\ 
}
%\setminted{%
%	breaklines,
%	breaksymbolleft=,      % vacía la marca de continuación
%	breaksymbolright=      % también limpia el símbolo a la derecha
%}


\lstdefinestyle{consola}{
	basicstyle=\footnotesize\bf\ttfamily,
	backgroundcolor=\color{gray75},
}	
\definecolor{gray}{rgb}{0.4,0.4,0.4}
\definecolor{darkblue}{rgb}{0.0,0.0,0.6}
\definecolor{cyan}{rgb}{0.0,0.6,0.6}
\lstset{language=XML}

\lstdefinelanguage{XML}{
	morestring=[b]",
	tabsize=2,
	breaklines=true,
	morestring=[s]{>}{<},
	morecomment=[s]{<?}{?>},
	stringstyle=\color{black},
	identifierstyle=\color{darkblue},
	keywordstyle=\color{cyan},
	numbers=left,
	morekeywords={xmlns,version,type}% list your attributes here
}

\lstdefinestyle{C}{language=C}
\lstdefinestyle{XML}{language=XML}

\DeclareMathOperator{\diag}{diag}

\newtcblisting{terminal}[2][]{
	listing engine=minted,
	listing only,
	#1,
	title=#2,
	minted language=bash,
	colback=mintedbackground,
	top=0mm,
	bottom=0mm
}

\newtcblisting{matlabcode}[2][]{%
  	listing engine=minted,
	listing only,
	title=#2,
	minted language=matlab,
	minted options={%
		linenos,            % activa numeración de líneas
		numbersep=1mm,      % separación texto–números
		breaklines,         % opcional: ajuste automático
		autogobble,          % opcional: recortar sangrías
		frame=leftline,     % línea vertical a la izquierda del bloque
		framesep=1mm,
		tabsize=4,
	},
%	colback=gray!5,
%	colframe=black!70,
	top=0mm,
	bottom=0mm
	#1
}


\newtcblisting{latexcode}[2][]{%
	listing engine=minted,
	listing only,
	title=#2,
	minted language=latex,
	minted options={%
		linenos,            % activa numeración de líneas
		numbersep=1mm,      % separación texto–números
		breaklines,         % opcional: ajuste automático
		autogobble,          % opcional: recortar sangrías
		frame=leftline,     % línea vertical a la izquierda del bloque
		framesep=1mm,
		tabsize=4,
	},
	%	colback=gray!5,
	%	colframe=black!70,
	top=0mm,
	bottom=0mm
	#1
}


\newtcblisting{consolestyle}[2][]{enhanced, listing engine=minted, 
	listing only,#1, title=#2, minted language=bash, 
	coltitle=mintedbackground!35!black, 
	fonttitle=\ttfamily\footnotesize,
	sharp corners, top=0mm, bottom=0mm,
	title code={\path[draw=mintedframe, dashed, fill=mintedbackground](title.south west)--(title.south east);},
	frame code={\path[draw=mintedframe, fill=mintedbackground](frame.south west) rectangle (frame.north east);}
}

\newenvironment{doble}
{\doublespacing
}

%\newcounter{comando}[section]
%\newenvironment{comando}[1][]{\refstepcounter{comando}\par\medskip
	%	\noindent \textbf{Comando~\thecomando. #1} \rmfamily}{\medskip}
%\begin{terminal}{#1}

%\end{terminal}
%}{\medskip}

\graphicspath{{img/}{tablas/}{portada/}}  % Las imágenes se buscarán en la carpeta "img"

\addto\captionsspanish{\renewcommand{\contentsname}{Índice}}

\def\CC{{C\nolinebreak[4]\hspace{-.05em}\raisebox{.4ex}{\tiny\bf ++ \space}}}
\def\Cc{{C\nolinebreak[4]\hspace{-.05em}\raisebox{.4ex}{\tiny\bf ++}}}
\newcommand{\ffolder}[1]{\menu{\faFolderO \space #1}}
\newcommand{\ffile}[1]{\menu{\faFileO \space #1}}
\newcommand{\folder}{\faFolderO \space}
\newcommand{\file}{\faFileO \space}
\newcommand{\world}{\faGlobe \space}
\newcommand{\wworld}[1]{\menu{\faGlobe \space #1}}
\newcommand{\SB}{\{B\}} % Define el sistema del cuerpo
\newcommand{\SI}{\{I\}} % Define el sistema inercial

\newcounter{actividad} % Define un contador llamado "actividad"

\newfloat{Comando}{h}{lot}[chapter]

\renewcommand{\tablename}{Tabla}
\renewcommand{\listtablename}{Índice de tablas}
\renewcommand\listingscaption{Código}
\newcommand{\code}[1]{\colorbox{lightgray}{\texttt{#1}}}

\renewcommand\lstlistingname{Código}
\renewcommand{\appendixname}{Anexo}
\renewcommand{\appendixtocname}{Anexos}
%\renewcommand{\appendixpagename}{Anexo}
\renewcommand\labelitemi{$\bullet$}